% Simple article, but with some useful ApJ things included
% Created by AYQH, 18 June 2016
\documentclass[12pt, letterpaper, preprint]{aastex}
\bibliographystyle{apj}
\usepackage{float, bm, graphicx, subfigure, amsmath, morefloats}
\usepackage{color}
\usepackage{tabularx}

% useful macros
\newcommand{\angstrom}{\mbox{\AA}}

\begin{document}

\title{VLASS Report}

\section{Basic Info \& Timeline}

VLASS has three epochs, 2017, 2019\dots
corresponding to 32-month cadence.
32 month cadence is important because that
lets the sky refresh totally.
The coverage is the entire sky above Socorro,
including the Galactic plane (which was missing
from previous surveys). 
It has wide bandwidth (S band, 2-4 GHz) enabling the
measurement of spectra.
The angular resolution (2.5 arcseconds) will be the highest
for some time to come, and enable precise localization.

2015: CNSS (VLASS pathfinder), 5 epochs
Summer 2016: pilot survey in the B configuration
Sept 2017: start observing.
In 2017 will do the first half of the sky,
probably concentrated in the northern sky, 17000 sq deg.
16 months later will do the other half,
then 16 months later will do the second epoch of the first half
So you get a snapshot of half the sky, separated by 16 months.
Pilot survey is 2 epochs (6 \& 7). real survey gets
an additional 3 epochs. 

Seven years, six VLA configuration cycles, total
time requirement is 5400 hours, 900 per configuration cycle

\section{Pilot survey}

S-band, OTF mosaicing, scanning in RA and constant Dec
Some areas will be covered with three passes to provide 
a similar sensitivity as that expected from three epochs 
of the full VLASS (70 microJy/beam) 
while others will be observed with a single pass (120 microJy/beam) 
to maximize sky coverage. 
The pilot will cover key galactic and extragalactic fields 
that have good multi-wavelength ancillary data, 
as well as covering areas of sky with good prior 
radio observations for technical validation of the OTF 
mosaicking observing mode. 
The total area to be covered will be ~2500 deg2, and will include
(for the fields with 3 passes, $70 \mu$Jy):

\begin{enumerate}
  \item Galactic plane fields: Galactic Center, Cygnus, Cepheus
  \item Extragalactic fields: Cosmological Evolution Survey, Sloane Digital Sky Survey (SDSS) Stripe 82, Chandra Deep Field South
\end{enumerate}

For the fields with 1 pass (120 $\mu$Jy):

\begin{enumerate}
  \item SDSS South Galactic Cap / FIRST southern sky for declination > 0 deg
  \item SDSS North Galactic Cap fields: Great Observatories Origins Deep Survey – North, Elais-N1, Lockman Hole, H-ATLAS North, Bootes
\end{enumerate}

\section{Key technological capabilities}

Rapid (near real-time data reduction)
Source extraction, transient identification pipelines
Follow-up
Systematic characterization of radio transient phase space
What to do with three epochs of a field 


\section{Brainstorming for the VLA Proposal}

\subsection{Why the VLA}

It will be complementary to the VLASS pilot.
We'll look at semi-random spots, have a snapshot not
locked in with the sky survey.
Fill in between sky survey epochs, few month
cadence picture. 
That said, the few month cadence is what we did for
Stripe-82\dots we'll do that sort of thing now for
a different area.

\begin{figure}[!p]
  \centering
  \includegraphics[scale=0.8]{../code/vla_proposal/coverage.png}
  \caption{The fields}
  \label{fig:coverage}
\end{figure}

\subsection{Why us}

We have all the infrastructure, 
data reduction pipeline (we’re also in charge of the VLASS)
We have the follow up capabilities
Our group has developed custom pipelines to remove RFI, 
calibrate data, image data, 
produce mosaics and extract source catalogs for ongoing 
VLA surveys for radio transients in the SDSS Stripe 82 
equatorial strip and the VLASS.
Extending these pipelines to LIGO follow up is
relatively straightforward, and actually we already
did it for O1 (ref Mooley et al. 2016). 


\end{document}


